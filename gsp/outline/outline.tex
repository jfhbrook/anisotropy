\documentclass[10pt, letterpaper]{article}
\begin{document}
    \section{Outline of Tasks}
    \begin{enumerate}
        \item Build and Run Numerical Model
        \begin{enumerate}
            \item Develop a geometrical model for FEA that makes sense
            \item Choose "good," reasonable values for physical properties in FE model
            \item Find a good method for automatically deriving \(k_\textrm{meas}\) from \((t,T)\) data at thermistor location given by FEM results
            \begin{enumerate}
                \item Find a good method for taking derivatives.
            \end{enumerate}
            \item Develop scripts and workflow to run all necessary models given a new set of parameters and retrieve useful results
        \end{enumerate}
        \item Interpret Results from Numerical Model
        \begin{enumerate}
            \item Find a good method for visualizing eigenvectors \& eigenvalues
            \begin{enumerate}
                \item Simple vectors in R3?
                \item Lame's ellipsoid?
                \item An adaptation of Mohr's Circle?
            \end{enumerate}
        \end{enumerate}
        \item Refine Numerical Model
        \begin{enumerate}
            \item Add cooling curve to numerical model
            \item Investigate inclusion of convective effects within snow
            \item Investigate more complex functions of \([K](x,y,z)\).
        \end{enumerate}
        \item Develop Solution to ``Reverse Problem''
        \begin{enumerate}
            \item I have some ideas about this, which I need to confirm as correct.
            \begin{enumerate}
                \item Find a mathematician to discuss some of these ideas with.
                \item Get said mathematician to join my thesis committee.
                \item Read more of the literature, to get a better background on the subject.
            \end{enumerate}
            \item Given that my ideas are correct, attempt to describe the inverse problem in a concise manner.
            \item Attempt to simplify the problem given a'priori knowledge of the snow geometry.
            \item Implement a numerical solution algorithm for the reverse problem.
            \item Benchmark the numerical solution as a function of \(k_\textrm{meas}\) results of FEM-based simulations against the input \([K]\) and angle of said simulations.
        \end{enumerate}
        \item Develop a Physical Test Procedure
        \begin{enumerate}
            \item Investigate Needle Selection and Construction
            \begin{enumerate}
                \item Make a possible attempt at building own needle probes
                \item Probably borrow needle probes from Dr. Sturm
            \end{enumerate}
            \item Investigate ``Snow'' Selection and Construction
            \begin{enumerate}
                \item Investigate prediction of aggregate properties of composites based on base materials and geometry
                \item Investigate materials with which such composite may be formed
                \item Construct composites
            \end{enumerate}
            \item Adapt Numerical Method for deriving \([K]\) to physical tests
        \end{enumerate}
        \item (Given Time) Address In-Situ Testing on Actual Snow (``Field measurements are non-trivial''--Jerry Johnson)
        \begin{enumerate}
            \item Predict difficulties in adapting test procedure to snow
            \item Adapt test procedure to meet these difficulties
        \end{enumerate}
        \item Address Questions Raised which Prompted this Research:
        \begin{enumerate}
            \item Can \(k_{zz}\) be predicted from \(k_{xy}\)? Does the gradient of \([k]\) affect this? Is directional anisotropy in real snow even significant? Can a simple correction factor be developed?
            \item How can we measure \([K]\) in snow? How many measurements would one need to do so? Can simplifying assumptions reduce the number of measurements? Can measurements be somehow reduced for a trade-off regarding accuracy?
            \item Needle probe measurements are consistently lower than similar measurements with a guarded hot-plate. Can this be (in part) explained by anisotropy in snow?
        \end{enumerate}
        \item Present Results to Public (such as the AGU Fall conference)
        \item Compile Results Into a Large Written Work Known as a Thesis
    \end{enumerate}
\end{document}
