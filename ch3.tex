\chapter{Numerical Needle Probe Approach}
\label{sec:numerical-np}
\bigskip

\section{Introduction} 
\label{sec:numerical-np:introduction}

Numerical experiments simulated needle probes in anisotropic mediums with
three-dimensional finite element heat transfer models in COMSOL 3.5a. While the
models themselves were relatively simple, attempting to automate a parameter
study in COMSOL 3.5a with respect to the anisotropic material properties proved
difficult.

\section{Geometry and Domain Properties}

\begin{table}[h]
\begin{tabular}
\label{tab:constants}
\caption{Constants Used in Numerical Models}
\end{tabular}{r | l}
\hline
radius of needle & \(0.25\) mm\\
length of needle & \(10\) cm\\
radius of snow & \(40\) cm\\
\hline
density of needle & \(8000\) kg/\(\textrm{m}^3\)\\
\(C_P\) of needle & \(460\) \textbf{*units*}\\
\(q\) of needle & \(0.5\) W/m\\
\(k\) of needle & \(160\) \textbf{*units*}\\
\hline
density of snow & \(200\) kg/\(\textrm{m}^3\)\\
\(C_P\) of snow & \(2050\) \textbf{*units*}
\end{tabular}
\end{table}

The needle was simulated as a long, thin steel cylinder embedded in the center
of a sphere of snow. While most of the dimensions and material properties were
held constant (see Table \ref{tab:constants}), the anisotropic conductivity of
the snow was parameterized in the form of a \(3\times3\) symmetric, positive
definite matrix.  In practice, this was done by specifying a diagonal matrix
\(\Lambda\) with positive eigenvalues \(k_{xy}\) and \(k_z\) and a rotation
matrix \(R\) around the \(x\) axis, and then defining \(K = R^T\Lambda R\) as in
equation \ref{eq:rotdiagrot} :

\begin{equation}
\label{eq:rotdiagrot}
K = \begin{bmatrix}
\cos(\theta) & 0 & \sin(\theta)\\
0 & 1 & 0\\
-\sin(\theta) & 0 &\cos(\theta)
\end{bmatrix}
\begin{bmatrix}
k_{xy} & 0 & 0\\
0 & k_{xy} & 0\\
0 & 0 & k_z
\end{bmatrix}
\begin{bmatrix}
\cos(\theta) & 0 & \sin(\theta)\\
0 & 1 & 0\\
-\sin(\theta) & 0 &\cos(\theta)
\end{bmatrix}
\end{equation}

\section{Geometry-Based Parameter Studies Using Comsol 3.5a}

% \section{Verification with Analytical Approach}
