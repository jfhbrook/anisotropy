A new method for measuring thermal conductivity in snow is being adapted from the method
of measuring isotropic thermal conductivity in snow with needle probes as developed by
Sturm and Johnson, in order to enable the determination of an anisotropic
thermal conductivity. \cite{sturm1,sturm2} This method has particular relevance to measuring thermal
conductivity of natural snowpacks where conductivity can be strongly anisotropic
due to structures that develop from vapor transport-induced metamorphism, self-compaction and other mechanisms.

Finite element numerical solutions are used to calculate the effective thermal
conductivity as a function of actual anisotropic thermal conductivity and needle
orientation. Values for anisotropic thermal conductivity are determined by
curve-fitting a function relating apparent results of the ASTM method at given
needle orientations and the actual solutions. The results of the numerical
simulations may then be applied to the measurement of actual anisotropic
materials. Unlike the isotropic case, multiple measurements at different
orientations are required.

% Need:
% * analytic solution
% * mention of What We Learned
